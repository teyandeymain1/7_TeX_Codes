\documentclass{ltjsarticle}
\usepackage[paperwidth=210mm, paperheight=297mm, margin=0.5pt]{geometry} % 余白をほぼゼロに設定
\usepackage{xeCJK} % For Japanese text; requires XeLaTeX or LuaLaTeX
\usepackage{amsmath, amssymb}
\usepackage{graphicx}
\usepackage{setspace}
\setstretch{0.5} 

\begin{document}

一般\\
- べき乗計算\\
$a^n$ について,適切な手法を用いると、計算量は二分探索と同じで$O(log n)$になる.\\

- do while文\\
whileループは条件式の後にブロック,do-whileループは条件式の前にブロックが来る(1回は必ず実行).\\

\\データ構造\\
- 二分探索木\\
最大深さ$d$, 各深さ $k(0≦k≦d)$ にちょうど$k+1$個のノードを持つ二分探索木の、木の深さをノードで表すと $n=k(k+1)/2 \leftrightarrow k=\sqrt{2n}$ となり、$O(n)$である\\

- 木と森\\
木 (tree): 連結な,(3ノード以上の単純)閉路を含まないグラフ
森 (forest): (3ノード以上の単純)閉路を含まないグラフ、連結でなくともよい.木の集まりと考えても良い)

- 平面的グラフ\\
平面的グラフ (planar graph): 辺の交差なしで平面に描画できるグラフ

-2連結成分分解\\
2連結のグラフとは,どの1点を消しても連結であるグラフのこと.
間接点は2連結成分分解において,2連結成分の境界となる点であり,深さ優先探索(DFS)を用いて求めることができる.
間接点は深さ優先探索木において2個以上の子を持つ.

\\ソート\\
- シェルソート\\
以下は1,5,3,6,7,8,0を1回シェルソートしたとき
$h=2$ のときは以下のように2段に分けて、各段でソートしていく\\
% シェルソートのグループAの過程を表で整理
\begin{tabular}{|c|c|c|c|c|c|c||c|c|c|c|c|c|c|}
\hline
1 &  & 3 &  & 7 &  & 0 & 0 &  & 1 &  & 3 &  & 7 \\
\hline
  & 5 &  & 6 &  & 8 &  & & 5 &  & 6 &  & 8 & \\
\hline
\end{tabular}
 結果: 0,5,1,6,3,8,7\\
\vspace{0.5ex}

\\探索\\
- 動的計画法\\
二つのノードの二項関係のみを考える $\rightarrow$ 最短経路\\
最長経路は二つ以上のノードを考える必要があるので独立でない\\

- トポロジカルソート\\
一筆書きのルートを探すパズル。トポロジカルソートでは,k個のノードに,1からkまでの番号(順序)iを付け全ての
有向辺(u,v)について,i(u) < i(v)である。iは,一般に,一通りに定まるか,複数ある.

- AVL木\\
バランス度の崩れるノード→左右の差が初めて2以上になるノードを探す\\

- ダイクストラ法\\
ダイクストラ法で、未確定のノード集合をヒープで管理するとヒープ生成に$Θ(mlogn)$必要なので、計算量が $O((|E|+|V|)log|V|)$ になる\\
















\end{document}