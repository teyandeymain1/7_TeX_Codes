\documentclass[aspectratio=169,12pt]{beamer}

\mode<handout>
{
\usepackage{pgfpages}
\pgfpagesuselayout{4 on 1}[a4paper,landscape]
}

\usefonttheme{professionalfonts}

\usepackage[english]{babel}
\usepackage[T1]{fontenc}
\usepackage{lmodern,bbding,pifont}
\usepackage{xspace}
\usepackage{transparent}
\usepackage{amsmath,amssymb,alltt,ifthen}
\usepackage[latin1]{inputenc}
\usepackage[english]{babel}
\usepackage{stmaryrd}
\usepackage{bussproofs}
\usepackage{transparent}

% ----------------------------Title----------------------------
\author{Yuta Sunagawa}
\begin{document}
  \title{Test} % Slide 1
  \institute{Ore University}
  \begin{frame}
    \maketitle
    \begin{quote}
    \end{quote}
  \end{frame}

% ----------------------------本体----------------------------
  % Slide 2
  \begin{frame}
    \frametitle{Goal and Motivation}
    \begin{alertblock}{Goal}
      {\underline{Goal:}}\\
      Explain unification problem and most general unifer (mgu).\\
      {\underline{Motivation:}}\\
        \begin{enumerate}
            \item Unification algorithm
          \begin{itemize}
            \item Input: unification problem $S$
            \item Output: most general unifier $\sigma$
          \end{itemize}
          \item Rewriting induction
        \end{enumerate}
    \end{alertblock}
    \begin{block}{Overview}<+->
      \begin{enumerate}
        \item What is unification problem? 
        \item Definition of most general unifer (mgu)
      \end{enumerate}
    \end{block}
  \end{frame}

  % Slide 3
  \begin{frame}
    \frametitle{Unification problem}
    \begin{definition}
      \alert{Unification problem} is a finite set of eqations of $S = \{s_1 =^? t_1, \ldots, s_n =^? t_n\}$.
    \end{definition}
    \begin{example}
      (copy from somewhere)
    \end{example}
  \end{frame}

  % Slide 4
  \begin{frame}
    \frametitle{Unifier}
    \begin{definition}
      \begin{itemize}
        \item  A substitution $\sigma$ is \alert{unifier} or \alert{solution} of $S$ if $\sigma s_i = \sigma t_i$ for $i = 1, \ldots, n.$
        \item $S$ is \alert{unifiable} if $\mathcal{U}(S) \ne \emptyset$ \quad ($\mathcal{U}(S)$ is set of all unifers of $S$).
      \end{itemize}
    \end{definition}
    \begin{example}
      (copy from somewhere)
    \end{example}
  \end{frame}

  % Slide 5
  \begin{frame}
    \frametitle{More general unifier}
    \begin{definition}
      $\sigma$ is more general than $\sigma'$ if there is a substitution $\delta$ such that $\sigma' = \delta\sigma$. \\
      In this case, we write $\sigma \lesssim \sigma'$.
    \end{definition}
    \begin{example}
      (copy from somewhere)
    \end{example}
    \begin{lemma}
      The relation $\lesssim$ on substitutions is a quasi-order.
    \end{lemma}
  \end{frame}

  % Slide 6
  \begin{frame}
    \frametitle{Most general unifier}
    \begin{definition}
      $\sigma$ is most general unifier (mgu) if $\sigma$ is $\mathcal{U}(S)$:
      \begin{itemize}
        \item $\sigma \in \mathcal{U}(S)$ and
        \item $\forall\sigma' \in \mathcal{U}(S)$.  $\sigma \lesssim \sigma'$.
      \end{itemize}
    \end{definition}
    \begin{example}
      (copy from somewhere)
    \end{example}
    \begin{theorem}
      If unification probrem \textit{$S$} has a solution then it has an mgu.
    \end{theorem}
  \end{frame}

  % Slide 7
  \begin{frame}

  \end{frame}

\end{document}