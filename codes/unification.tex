\documentclass[aspectratio=169,12pt]{beamer}

\mode<handout>
{
\usepackage{pgfpages}
\pgfpagesuselayout{4 on 1}[a4paper,landscape]
}

\usefonttheme{professionalfonts}

\usepackage[english]{babel}
\usepackage[T1]{fontenc}
\usepackage{lmodern,bbding,pifont}
\usepackage{xspace}
\usepackage{transparent}
\usepackage{amsmath,amssymb,alltt,ifthen}
\usepackage[latin1]{inputenc}
\usepackage[english]{babel}
\usepackage{stmaryrd}
\usepackage{bussproofs}
\usepackage{transparent}

% for highlighting texts (background)
\newcommand<>\hl[2]{%
  \alt#3{\makebox[\dimexpr\width-2\fboxsep]{\smash{\colorbox{#1}{$#2$}}}}{#2}%
}

\usepackage{tikz}
\usetikzlibrary{arrows,positioning,decorations}

% color scheme
\definecolor{green}{rgb}{0.0,0.5,0.0}
\definecolor{blue}{rgb}{0.0,0.0,0.7}
\definecolor{red}{rgb}{0.8,0.0,0.0}
\definecolor{lightred}{rgb}{1.0,0.97,0.97}
\definecolor{lightblue}{rgb}{0.95,0.95,1.0}
\definecolor{darkblue}{rgb}{0.3,0.3,0.5}
\definecolor{darkred}{rgb}{0.6,0.0,0.0}
\definecolor{darkorange}{rgb}{0.6,0.2,0.1}
\definecolor{darkgreen}{rgb}{0,0.3,0}
%
\newcommand<>{\caution}[1]{\textcolor#2{magenta}{#1}}
\newcommand<>{\cautionblue}[1]{\textcolor#2{blue}{#1}}
\newcommand<>{\strong}[1]{\textcolor#2{blue}{#1}}
\newcommand<>{\STRONG}[1]{\textcolor#2{blue}{\bfseries #1}}
\newcommand<>{\CAUTION}[1]{\textcolor#2{magenta}{\bfseries #1}}
\newcommand<>{\ALERT}[1]{\alert#2{\bfseries #1}}
\newcommand{\IMAGE}[2]{\pgfdeclareimage[#1]{#2}{#2}\pgfuseimage{#2}}

\setbeamerfont{title}{series=\bfseries,size=\Large}
\setbeamercolor{title}{fg=red}
\setbeamerfont{subtitle}{series=\bfseries,size=\LARGE}
\setbeamercolor{subtitle}{fg=red}
\setbeamerfont{author}{series=\bfseries,size=\large}
\setbeamercolor{author}{fg=blue}
\setbeamerfont{institute}{series=\bfseries,size=\large}
\setbeamercolor{institute}{fg=green}

\setbeamertemplate{blocks}[rounded]
\setbeamerfont{block title}{series=\bfseries,family=\sffamily,size=\small\strut}
\setbeamercolor{block title}{bg=darkblue,fg=white}
\setbeamercolor{block body}{bg=lightblue}
%\setbeamercolor{block title}{bg=white,fg=darkblue}
%\setbeamercolor{block body}{bg=white}
\setbeamercolor{block title alerted}{bg=darkred,fg=white}
\setbeamercolor{block body alerted}{bg=lightred}
\setbeamercolor{structure}{fg=darkred}
% no fading effect
\makeatletter
\pgfdeclareverticalshading[lower.bg,upper.bg]{bmb@transition}{200cm}{%
  color(0pt)=(lower.bg); color(4pt)=(lower.bg); color(4pt)=(upper.bg)}
\makeatother

% frametitle
\useframetitletemplate{
\begin{centering}
\centerline{\large\bfseries\color{darkblue}\insertframetitle}
\end{centering}
}

% footer  (title  page/pages)
\setbeamertemplate{navigation symbols}{}
\useheadtemplate{\vbox{\vskip8pt}}
\usefoottemplate{\vbox{\vskip2pt\inserttitle\hfil\insertframenumber/\inserttotalframenumber\vskip5pt}}


% enumerate/itemize environment
\newcommand*\tikzboxed[1]{\tikz[baseline=(c.base)]{%
\node[thick,shape=rectangle,draw,inner sep=2pt] (c) {#1};}}
\setbeamertemplate{items}[square]
\setbeamertemplate{enumerate item}{\tikzboxed{\footnotesize\insertenumlabel}}
\setlength{\itemsep}{1ex}

% colored underline, with Beamer overlay support
% usage: \cul{x} or \cul[blue]{x} or \cul<2->{x} or \cul<2->[blue]{x}
\newcommand<>{\cul}[2][red]{
\fontdimen8\textfont3=0.75pt%
\alt#3%
{\color{#1}\underline{{\color{black}#2}}\color{black}}%
{\transparent{0.0}\underline{{\transparent{1.0}#2}}\transparent{1.0}}%
}

\newcommand{\semitransp}[2][35]{\textcolor{fg!#1}{#2}}

\newcommand{\seq}[2][n]{{#2_1},\dots,{#2_{#1}}}

\newcommand{\m}[1]{\text{#1}}
\renewcommand{\AA}{\mathcal{A}}
\newcommand{\BB}{\mathcal{B}}
\newcommand{\EE}{\mathcal{E}}
\newcommand{\NN}{\mathbb{N}}
\newcommand{\FF}{\mathcal{F}}
\newcommand{\GG}{\mathcal{G}}
\newcommand{\PP}{\mathcal{P}}
\newcommand{\RR}{\mathcal{R}}
\newcommand{\TT}{\mathcal{T}}
\newcommand{\VV}{\mathcal{V}}
\newcommand{\ZZ}{\mathbb{Z}}
\def\tuple<#1>{\langle #1 \rangle}
\renewcommand{\emptyset}{\varnothing}
\newcommand{\Pow}[1]{\text{Pow}(#1)}
\newcommand{\NF}{\text{NF}}

\renewcommand{\arraystretch}{1.5}

% ----------------------------due to Aart----------------------------
\newcommand{\mirror}[1]{\reflectbox{${#1}$}}
\newcommand{\instance}{\mathrel{\makebox[0pt]{\makebox[8pt][r]%
{\raise 1pt \hbox{$\cdot$}}}{\geq}}}
\newcommand{\Instance}{\gtrdot}
\newcommand{\subsumes}{\mathrel{\mirror{\instance}}}
\newcommand{\Subsumes}{\mathrel{\mirror{\gtrdot}}}

\newcommand{\litsim}{\doteq}


\newcommand{\Dom}{\mathcal{D}\mathsf{om}}
\newcommand{\II}{\mathcal{I}} % introduced variables.

\newcommand{\lpo}{\mathsf{lpo}}
\newcommand{\lex}{\mathsf{lex}}

\newcommand{\angled}[1]{\langle #1 \rangle}

\newtheorem{prop}{Proposition}
% \newtheorem{corollary}{Corollary}

% ----------------------------Title----------------------------
\author{Yuta Sunagawa}
\begin{document}
  \title{Unification Problem} % Slide 1
  \institute{JAIST}
  \begin{frame}
    \maketitle
    \begin{quote}
    \end{quote}
  \end{frame}

% ----------------------------本体----------------------------
  % Slide 2
  \begin{frame}
    \frametitle{Goal and Motivation}
    \begin{alertblock}{Goal}
      {\underline{Goal:}}\\
      Explain unification problem and most general unifer (mgu).\\
      {\underline{Motivation:}}\\
        \begin{enumerate}
            \item Unification algorithm
          \begin{itemize}
            \item Input: unification problem $S$
            \item Output: most general unifier $\sigma$
          \end{itemize}
          \item Rewriting induction
        \end{enumerate}
    \end{alertblock}
    \begin{block}{Overview}<+->
      \begin{enumerate}
        \item What is unification problem? 
        \item Definition of most general unifer (mgu)
      \end{enumerate}
    \end{block}
  \end{frame}

  % Slide 3
  \begin{frame}
    \frametitle{Unification problem}
    \begin{definition}
      \alert{Unification problem} is a finite set of eqations of $S = \{s_1 =^? t_1, \ldots, s_n =^? t_n\}$.
    \end{definition}
    \begin{example}
      (copy from somewhere)
    \end{example}
  \end{frame}

  % Slide 4
  \begin{frame}
    \frametitle{Unifier}
    \begin{definition}
      \begin{itemize}
        \item  A substitution $\sigma$ is \alert{unifier} or \alert{solution} of $S$ if $\sigma s_i = \sigma t_i$ for $i = 1, \ldots, n.$
        \item $S$ is \alert{unifiable} if $\mathcal{U}(S) \ne \emptyset$ \quad ($\mathcal{U}(S)$ is set of all unifers of $S$).
      \end{itemize}
    \end{definition}
    \begin{example}
      (copy from somewhere)
    \end{example}
  \end{frame}

  % Slide 5
  \begin{frame}
    \frametitle{More general unifier}
    \begin{definition}
      $\sigma$ is more general than $\sigma'$ if there is a substitution $\delta$ such that $\sigma' = \delta\sigma$. \\
      In this case, we write $\sigma \lesssim \sigma'$.
    \end{definition}
    \begin{example}
      (copy from somewhere)
    \end{example}
    \begin{lemma}
      The relation $\lesssim$ on substitutions is a quasi-order.
    \end{lemma}
  \end{frame}

  % Slide 6
  \begin{frame}
    \frametitle{Most general unifier}
    \begin{definition}
      $\sigma$ is most general unifier (mgu) if $\sigma$ is $\mathcal{U}(S)$:
      \begin{itemize}
        \item $\sigma \in \mathcal{U}(S)$ and
        \item $\forall\sigma' \in \mathcal{U}(S)$.  $\sigma \lesssim \sigma'$.
      \end{itemize}
    \end{definition}
    \begin{example}
      (copy from somewhere)
    \end{example}
    \begin{theorem}
      If unification probrem \textit{$S$} has a solution then it has an mgu.
    \end{theorem}
  \end{frame}

  % Slide 7
  \begin{frame}

  \end{frame}

\end{document}